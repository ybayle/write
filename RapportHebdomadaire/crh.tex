\documentclass[12pt, a4paper]{article}
\usepackage[utf8]{inputenc}
\usepackage{natbib}
\usepackage[french]{babel}
\usepackage[T1]{fontenc}
\usepackage{graphicx}
\usepackage{fancyhdr}
\usepackage{lastpage}
\usepackage[margin=2cm, headheight=2cm]{geometry}
\pagestyle{fancy}
\usepackage{hyperref}
\fancyhf{}
\lhead{Prénom Nom}
\chead{Compte rendu hebdomadaire du mercredi 15 juillet 2020}
\rhead{\thepage / \pageref{LastPage}}
\AtBeginDocument{\def\labelitemi{$\bullet$}} % textbullet textendash textasteriskcentered textperiodcentered blacksquare square bullet circ
\AtBeginDocument{\def\labelitemii{$\circ$}}
\begin{document}
\section{Travaux effectués}

\begin{itemize}
    \item Description tâche 1 avec comme sous-tâches:
          \begin{enumerate}
              \item Sous-tâche 1
              \item Sous-tâche 2
              \item Sous-tâche 3 projet ARMS
          \end{enumerate}
    \item Description tâche 2 avec l'\autoref{eq:deconvolution}

          \begin{equation}
              \label{eq:deconvolution}
              h(n) = IFFT \left( \frac{FFT(y(n))}{FFT(x(n))} \right)
          \end{equation}

\end{itemize}

\section{Travaux en cours}

\begin{itemize}
    \item Description tâche 1 en lien avec la \autoref{fig:logo}

          \begin{figure}[h!]
              \centering
              \includegraphics[width=\linewidth]{../img/logo}
              \caption{Exemple de titre de figure}
              \label{fig:logo}
          \end{figure}

    \item Voilà la Description de ma tâche numéro 2

\end{itemize}

\section{Travaux à réaliser}

\begin{itemize}
    \item Description tâche 1 en lien avec le \autoref{table:performances}

          \begin{table}[!h]
              \centering
              \caption{Exemple de titre de tableau}
              \label{table:performances}
              \begin{tabular}{c|ccc}
                           & \multicolumn{3}{c}{(x,y)}                             \\
                  Position & cm                        & \%            & Float     \\
                  \hline
                  A        & ( 0 cm,  0 cm)            & (  0\%,  0\%) & (0.0,0.0) \\
                  B        & (13 cm,  0 cm)            & (100\%,  0\%) & (1.0,0.0) \\
                  C        & ( 0 cm,240 cm)            & (  0\%,100\%) & (0.0,1.0) \\
                  D        & (80 cm,240 cm)            & (100\%,100\%) & (1.0,1.0) \\
              \end{tabular}
          \end{table}

    \item Description tâche 2 comme \citep{Zhang2018LSTM}
\end{itemize}

\bibliographystyle{plainnat}
\bibliography{references}
\end{document}
