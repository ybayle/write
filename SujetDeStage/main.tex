\documentclass[12pt, a4paper]{article}
\usepackage{graphicx}               % Display images
\usepackage[T1]{fontenc}            % Encoding
\usepackage[french]{babel}          % French dict
\usepackage[margin=2cm]{geometry} % Update margin
\usepackage{hyperref}               % \url{https://www.site.com/fr/} use [backref] to have citation go back to citing text
\hypersetup{colorlinks = true,      % Colours links instead of ugly boxes
            urlcolor   = blue,      % Colour for external hyperlinks
            linkcolor  = blue}      % Colour of internal links
\setlength\parindent{0pt}           % Remove indentation at the beginning of paragraph
\newcommand\todo[1]{\colorbox{red}{\textcolor{blue}{[TODO: #1]}}} % displays a to do note
\begin{document}                    % Début du document
\thispagestyle{empty}               % Remove page numbering
\begin{center}
    \includegraphics[width=5cm]{../img/logo} \hspace{1cm} \includegraphics[width=5cm]{../img/logo} ~\\~\\~\\~\\
    \section*{Offre de stage en informatique niveau Bac\texttt{+}5}
\end{center}

~\\

\subsection*{\todo{Nom du projet}}

Vous serez intégré(e) à l'équipe du département R\&D et participerez à l'avancée du projet \todo{Nom du projet}.
Dans un premier temps, vous appréhenderez l'état de l'art afférent aux \todo{Domaine}, à la simulation en temps réel et aux contraintes audio professionnelles.
Vous prendrez ensuite en main les outils existants externes et internes de \todo{Outil} en Python afin d'ajouter de nouvelles fonctionnalités.
La dernière partie du projet consistera à écouter et valider \todo{Produit} afin d'améliorer \todo{Méthode}.

\subsection*{Profil recherché}

Nous recherchons un(e) élève ingénieur(e) en informatique passionné(e) par le monde de l'audio et de la musique.
Le projet comportera de la programmation en Python 3 et en C++17 avec JUCE.
La mission demande des capacités d'organisation, de la rigueur, de l'autonomie et le goût du travail en équipe.

\subsection*{Entreprise}

L'entreprise ont été créées en XXXX et est située à XXXX.

\subsection*{Durée}

Six mois, à commencer entre Janvier et Mars 2020.

\subsection*{Lieu}

Adresse

\subsection*{Renseignements et candidature}

Envoyer votre CV et votre lettre de motivation à XXXX.

\end{document}
